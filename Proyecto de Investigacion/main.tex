\documentclass{article}
\usepackage[utf8]{inputenc}

\title{Proyecto de investigacion}
\author{Jose David Ortiz Miranda}
\date{March 2020}

\begin{document}

\maketitle

Las matemáticas eran infalibles, tenían en sus fórmulas, ecuaciones y en sus teoremas respuestas a la mayoría de preguntas que la humanidad tenía en ese momento, fue hasta finales del siglo XIX y principios del siglo XX, que se empieza a explorar problemas que implican procesos extensos, que conllevan operaciones sucesivas y en esa sucesión, a veces, implicaba trabajar con números extremadamente grandes, para esos problemas las matemáticas aun no tenían una respuesta que no fallara, y se volvía un proceso muy tedioso para hacerlo solo con papel y lápiz. Un ejemplo de esto es claramente la paradoja de Russell, que genero un impacto en el mundo de las matemáticas porque era un potente detonante a la base que los matemáticos habían generado, toda esta coyuntura derivo en la llamada “crisis de los fundamentos”, que llevaría a una conclusión, que las matemáticas fallaban en ciertas ocasiones.\\
Años antes de que estallara la crisis de los fundamentos, la comunidad científica se dividió en varios movimientos, con ideas distintas, pero los formalistas, uno de los grupos que surgió con esta coyuntura, se aferraba a la idea de que todo era alcanzable por medio de las matemáticas, dedicándole un tiempo prudente, este movimiento era liderado por el reconocido matemático David Hilbert, quien lidero la llamada misión “Programa de Hilbert”, esta misión tenia el fin de demostrar que los sistemas axiomáticos bien definidos tenía tres propiedades que los convertían en infalibles, estas eran, los axiomas deben ser consistentes, esto quiere decir que no permitían contradicciones, también debían ser finitarios quiere decir que se podían llevar a cabo con un conjunto de pasos, precisos y lógicos los cuales terminaban en algún momento, y como ultimo requisito que los axiomas eran completos, es decir, se podía demostrar que una afirmación era cierta o que era falsa. Fue un joven  austriaco llamado Kurt Gödel que en un congreso matemático que se llevo a cabo en la ciudad de Konigsberg, que afirmo que estaba a punto de completar una demostración que ponía fin a la discusión y que probaba que ningún sistema podía cumplir los requisitos del programa de Hilbert, Gödel un año después de este hecho, publico su Primer teorema de Incompletitud en el cual explicaba de una forma muy compleja y minuciosa que si un sistema esta definido de manera que no existan contradicciones existirán enunciados en el que no se podrán demostrar su falsedad o veracidad.\\
Con estas ideas ya en el mundo matemático, se añadieron matemáticos que se encargaron de llevar el legado de Gödel, uno de ellos y de los mas relevantes es Alan Turing que antes de su colaboración en la segunda guerra mundial con su máquina “Enigma”, había publicado un artículo que mostro que existen los problemas no resolubles y que no se podía saber cuales son estos problemas. Este utilizo un razonamiento parecido al de Gödel, pero Turing por su parte afirmó que no es posible determinar si un problema escogido al azar tiene solución o por el contrario no la tiene.\\
Con base en esto Alan Turing, demostró que hay problemas que no se pueden computar, para lograr esto se baso en establecer una noción muy detallada de computación efectiva, basada en la idea de su máquina universal.\\
Esta máquina era un dispositivo muy sencillo que constaba de una cinta de papel infinita dividida por casillas, tiene una cabeza que es capaz de leer y sobrescribir símbolos en las casillas y mover la cinta hacia la derecha o la izquierda, y tiene una serie de instrucciones que constituyen el “programa” de la máquina. Turing también demostró que esta máquina o un conjunto de ellas podían realizar tareas con un conjunto de pasos a seguir.\\
Este es el principio de funcionamiento de cualquier dispositivo “inteligente” que se tenga a la mano, puede ser un computador o dispositivo móvil smartphone, hoy en día nos parece muy cotidiano el utilizar alguno de estos dispositivos, incluso varios para realizar cualquier actividad académica o laboral que se nos proponga, pero históricamente podemos ver que fue un proceso con muchos tropiezos y con mentes muy prodigiosas que se esforzaron para desarrollar cosas que tal vez no se imaginarían hasta donde llevaran a la humanidad.\\
Hoy en día estos dispositivos han cambiado mucho en su aspecto y en su eficiencia, pero su principio de funcionamiento sigue siendo el mismo, son máquinas universales a pequeña escala que realizan una serie de instrucciones que nos permiten comunicarnos, ver una foto de un familiar en el extranjero, o recibir formación académica a distancia, incluso laboral a distancia, no es secreto que estas máquinas nos facilitan mucho las cosas en nuestro día a día, además se han convertido en un implemento fundamental para relacionarnos actualmente.\\
Como es de esperarse la humanidad sigue con su curiosidad y con su ambición de llegar más allá, y este ámbito no es la excepción, los últimos 30 años han venido desarrollando máquinas cada vez más eficientes, livianas, estéticamente mejores, que se pueden llevar de un lugar a otro con facilidad, y que nos permiten estar en contacto con cualquier persona en el mundo. Los avances son a pasos agigantados y cuando se cree que no se puede avanzar más, nos sorprenden con dispositivos altamente veloces, eficientes, con gran capacidad para realizar multiples tareas al mismo tiempo, lo que si es cierto es que actualmente la utilización de alguno de estos dispositivos es cosa de todos los días y que nos esperan cosas que nos sorprenderán cada día más, y todo empezó con problemas matemáticos y con paradojas que las matemáticas no podían resolver pasando por máquinas capaces de realizar tareas algorítmicas y derivando en la computación.



\end{document}
